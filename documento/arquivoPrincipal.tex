% Documento de demonstração do repositório leleobhz/latex-github. Baseado fortemente em https://www.ime.usp.br/~viviane/MAP2212/exemplo.tex
% Seria 202204051707

%Tudo que começa com '%' é comentário e é ignorado pelo compilador

%Gerando arquivo em latex:
%latex arquivo.tex (em dvi)
%pdflatex arquivo (em pdf)
%dvipdfm arquivo
%s2pdf arquivo

% Alguns modos de usar o latex:
% Windows – Miktex com Led
% Linux – texlive com kile

\documentclass[12pt,a4paper]{report} %aqui fala o tipo de documento e o tamanho da fonte. Opções: tamanho do texto (10pt, 12pt, 14pt), formato do papel (a4paper, a5paper, b5paper, letterpaper, legalpaper, executivepaper), o número de colunas (onecolumn, twocolumn), entre outras opções.
%Por exemplo, [12pt,a4,twocolumn].
%classe: article, report, letter, book ou slides. Instalar abnt para quem está pensando no tf
\usepackage[brazil]{babel} %hifenização em português do brasil
% Nota do autor do repo leleobhz/latex-github: UTF-8 é melhor, convenhamos. A linha abaixo é comentada devido ao uso do lualatex, que só aceita UTF-8
%\usepackage[cp1252]{inputenc} %Encoding do windows, com caracteres acentuados 
\usepackage{fontspec}
\usepackage{mathptmx}
\usepackage{amsmath}
\usepackage{amssymb} %alguns caracteres matemáticos especiais
\usepackage{graphicx} %Para inserir figuras

% Aqui começa o documento
\begin{document}
\title{Exemplo de \LaTeX} % título
\author{Viviane Teles de Lucca Maranhão \\ viviane@ime.usp.br} % quem escreveu
\date{março de 2012} %data

\maketitle %cria o título

\def \negritovi {\textbf} %Criando comandos

\tableofcontents %índice
\pagebreak % Quebra de página
\listoffigures %indice de figuras
\listoftables %indice de tabelas
\pagebreak % Quebra de página

\section{Introdução} %Cria uma seção
\mbox{} %As vezes, para o primeiro parágrafo da seção sair com espaçamento correto precisa dessa gambiarrinha


Este é um tutorial bem simples de \LaTeX. Existe muito material disponível na internet, inclusive na página do professor em \cite{stern}.
% para pular um paragrafo basta pular uma linha

Este é o segundo parágrafo.
\\ Passando para a linha de baixo. %Aqui inicia uma nova linha, mas sem o espaçamento.

Os comandos em \LaTeX começam com $\backslash$ %Esta é a \. 
\\Outros exemplos de caracteres especiais são \# \$ \%  \& \_ \{ \} que podem ser escritos com um ' $\backslash$ ' na frente.

\textbf{Um texto em negrito} %Negrito
\\ \negritovi{Outro negrito} % aquele comando novo criado no começo
\\ \textit{Um texto em itálico} %Itálico




\section{Itens}
\mbox{}


Escrevendo em itens:

\begin{itemize} %Começa uma lista
\item Primeiro item %cada item é acrescentado com '\item'
\item Segundo item
\item ...
\end{itemize} % termina a lista

\section{Equações e caracteres matemáticos}

\mbox{}

Alguns símbolos matemáticos:

$\pi $ % pi
$\Pi $ % Pi maiúsculo
$\frac{numerador}{denominador} $ % Fração

\^ % Sobrescrito
\_ %Subscrito

$\backslash $ % \
$\sqrt{num}$  %Raiz quadrada do numero
$\int $ %Integral
$\sum_{i = 1}^{n} $ % Somatoria de i = 1 a n
$\leq $ %Menor ou igual
$\geq $ % Maior ou igual
$\neq $ % Diferente
$\sim $ % Semelhante ~
$\approx $ % Aproximadamente 
$\sin $ %seno  

Uma fórmula pode ser iniciada e terminada com \$. Isso faz com que ela seja escrita na mesma linha que o texto. Por exemplo: $\pi \approx 3.1415 $.

Podemos também destacar uma fórmula iniciando e terminado com \$\$. Por exemplo:

$$ \frac{1}{2} = 0.5 $$ %frações são usadas assim, primeiro par de chaves é o numerador e o segundo o denominador.

Por fim, podemos criar um ambiente de equação, dado um número a ela para que ela possa ser referenciada posteriormente:

\begin{equation}\label{equacao}
A^{i}_{j} = B^{j}_{i} % brincando com índices
\end{equation} % Formula numerada


\section{Tabela}

\mbox{}

Uma tabela:


\begin{table}[ht] % começa a tabela
\begin{center} %centraliza
\begin{tiny} %quero o tamanho da fonte bem pequeno nessa tabela
\begin{tabular}{rrl} %especifica a quantidade de linhas e alinhamento
  \hline %linha horizontal
Direita & Direita  &Esquerda \\
%separa coluna com '&' e quebra linha com '\\'
  \hline
1 & 2 & 3 \\
4 & 5 & 6 \\
   \hline
\end{tabular}
\end{tiny} %volta pro tamanho original
\caption{Uma tabela simples} %Legenda
\label{tabelinha} %nome
\end{center} %fecha o centralizado
\end{table} %termina a tabela

\section{Figuras}
\mbox{}

Aqui vai uma figura:

\begin{figure}[h!t] %Aqui, senão no top
\centering % centralizada
\includegraphics[scale = 2]{figura.jpg} %inclui figura.png com escala 2
\caption{Uma figura qualquer} %legenda
\label{figurinha} % nome da figura
\end{figure}

\cleardoublepage %faz flush do buffer, pode ser util para tentar colocar figuras e tabelas onde realmente desejamos

\begin{thebibliography}{99} % Até 99 referencias
\bibitem{wiki} % este é o nome que é usado para citar a referencia
http://en.wikipedia.org/ % referência
\bibitem{stern}
http://www.ime.usp.br/$\sim$ jstern

\end{thebibliography}

\end{document}

%fim do documento. O que vem depois daqui não é gerado

\section{Escondida}
Essa seção não vai aparecer.
